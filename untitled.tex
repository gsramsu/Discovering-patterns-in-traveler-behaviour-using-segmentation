\begin{abstract}
Current revenue management practice in most airlines involves demand segmentation through artificial pricing rules on RBDs rather than through a more precise quantification of traveler behavior prototypes. We consider the problem of finding these common behavioral patterns among travelers in an airline network through the process of clustering. We begin by characterizing travel in terms of a number of features that pertain to booking and travel behavior as well as preferences with respect to price and schedule sensitivity, inasmuch as they can be understood from the behavior itself. Trips thus characterized are then grouped using an ensemble clustering algorithm that aims to find stable clusters as well as discover subgroup structures within groups. A multidimensional analysis of traveler behavior based on these groupings leads us to discover non-trivial patterns that can then be exploited for better revenue management. The approach also has relevance in the currently changing world of airline reservations, in that it provides airlines with behavioral templates upon which to base branded fare strategies and leverage the capabilities that IATA’s NDC standard has to offer.
\end{abstract}

\section{Background and motivation}
\label{sec:intro}

In this section, we motivate the paper by providing some context to why segmentation matters, what we want to accomplish and what has currently been done. Literature survey will be a subsection here, so that the context of this work is set.

\subsection{Literature survey}
\label{sec:lit}

This section will contain references in 3 areas: 

\begin{enumerate}
\item Clustering algorithms
\item Segmentation in airlines and other domains and how it has been used
\item How airlines currently use traveler behaviour data, insofar as they do use it
\end{enumerate}

The objective here is to provide a reasonably clear picture of what the gap is, so that this paper can then address it.

\section{Methodology}
\label{sec:method}

In this section, we provide a broad description of the whole pipeline, with specific subsections to discuss the details of steps in the pipeline.

\subsection{Data sources}
\label{sec:data}

What kind of data do we use?

\subsection{Feature engineering}
\label{sec:features}

How do we characterize a PNR?

\subsection{Clustering PNRs to discover segments}
\label{sec:clusters}

How do we discover segments in the data using clustering? Describe our overall approach (ensemble method used as a wrapper), and the specific algorithms tried out.

\subsection{Discovering patterns in traveler behaviour}
\label{sec:patterns}

How do we discover patterns in the data? Describe the Hellinger Distance based approach here, but lay the foundations by presenting a framework for pattern detection.

\section{Illustrative results on airline data}
\label{sec:results}

Describe results on airline data in a depersonalized form.

\section{Scope for further work}
\label{sec:concl}

What else can we do?


  
  